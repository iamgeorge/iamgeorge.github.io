\documentclass[12pt]{article}
\usepackage[utf8]{inputenc}
\usepackage{enumitem}
\usepackage{tabularx}
\usepackage{hyperref}

\title{English Corner - Lesson Plan}
\author{George}
\date{2023 Summer}

\begin{document}

\maketitle

\section{Class Details}
\begin{itemize}
    \item \textbf{Course Name:} English Corner
    \item \textbf{Lesson Topic:}  First class - introduce yourself and your name 
    \item \textbf{Date:} Monday, July 31, 2023
    \item \textbf{Time:} 9:50am - 10:30am (45 minutes)
\end{itemize}

\section{Objectives}
By the end of this lesson, students should be able to:
\begin{itemize}
    \item Have an English name
    \item Know how to introduce themself

\end{itemize}

\section{Materials Required}
\begin{itemize}
   
    \item list of prompts to make names

  
\end{itemize}


\section{Students}
% List the students here
\begin{itemize}
    \item 

\end{itemize}
\section*{Introduction and English Name Assignment (10 minutes)}

\begin{enumerate}[leftmargin=*]
\item \textbf{Minute 1-2:} Start the class by introducing yourself in English and Chinese (if you can). Explain that you will be their English teacher and that you are excited to learn with them. 

\item \textbf{Minute 3-4:} Explain that everyone will be given an English name. This is a common practice in ESL classrooms and can help students feel more comfortable speaking English. 

\item \textbf{Minute 5:} Ask the first student to come to the front of the class. 

\item \textbf{Minute 6:} Ask the student to share something about themselves in Chinese - it could be their favorite hobby, animal, color, or anything else they feel comfortable sharing. 

\item \textbf{Minute 7:} Based on their prompt, assign them an English name that somehow relates to what they shared. For example, if a student says they love the color blue, you might name them "Sky". If a student says they love tigers, you might name them "Tiger". 

\item \textbf{Minute 8:} Explain why you chose the name, so the student understands the connection between their prompt and their new English name. 

\item \textbf{Minute 9-10:} Repeat the process with the next student. 
\end{enumerate}



\section{Future Plan}
% List the students here
\begin{itemize}
    \item 

\end{itemize}
\section{Plan Completion}
% How much of the plan was completed?
\begin{itemize}
    \item 
\end{itemize}
Left for next class: 
\begin{itemize}
    \item 
\end{itemize}
\end{document}
