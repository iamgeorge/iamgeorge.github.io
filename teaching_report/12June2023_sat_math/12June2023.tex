\documentclass[12pt]{article}
\usepackage[utf8]{inputenc}
\usepackage{enumitem}
\usepackage{tabularx}
\usepackage{hyperref}

\title{SAT Math - Lesson Plan}
\author{George}
\date{2023 Summer}

\begin{document}

\maketitle

\section{Class Details}
\begin{itemize}
    \item \textbf{Course Name:} SAT Math
    \item \textbf{Lesson Topic:}  basic trignomitry, practice and trig of any angles
    \item \textbf{Date:} Wednesday, July 12, 2023
    \item \textbf{Time:} 9:00am - 10:30am (90 minutes)
\end{itemize}

\section{Objectives}
By the end of this lesson, students should be able to:
\begin{itemize}
    \item Understand sin, cosine, tangent
    \item Know how to appy trig to solve problems
    \item Use the trigonometry of any angles
\end{itemize}

\section{Materials Required}
\begin{itemize}
   
    \item CLASS 10, INTRO TO TRIGONOMETRY.ppt
    \item class Practice: sat 2022-03-us 3.17,4.12,4.36(x-intercept)
    \item class Practice: sat 2022-03-as 3.6(tan)
  
\end{itemize}


\section{Students}
% List the students here
\begin{itemize}
    \item Peter

\end{itemize}
\section{Future Plan}
% List the students here
\begin{itemize}
    \item exponantial function
    \item exponantial rules
    \item congruent shapes
    \item box-whisker plot

\end{itemize}
\section{Plan Completion}
% How much of the plan was completed?
\begin{itemize}
    \item Understand and explain the concept of paralell and perpendicular lines
    \item Apply the relationship between the lines with slope
\end{itemize}
Left for next class: 
\begin{itemize}
    \item Understand sin, cosine, tangent
\end{itemize}
\end{document}
