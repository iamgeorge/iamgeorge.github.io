\documentclass{article}
\usepackage{amsmath}

\begin{document}

\section*{Simple Interest Questions}

\subsection*{Questions}
\begin{enumerate}
  \item Maria borrowed \$500 from a friend and agreed to pay simple interest at an annual rate of 4\%. If she pays a total of \$60 in interest, how long was the loan outstanding?
  \vspace{2cm}
  
  \item A savings account offers a simple interest rate of 5\% per year. If John deposits \$1,000, how much interest will he have earned after 3 years?
  \vspace{2cm}
  
  \item A bond with a face value of \$1,000 is issued with a simple interest rate of 6\% per year. After 5 years, the bond has earned \$300 in interest. Was the bond sold at face value, at a discount, or at a premium, and by how much?
  \vspace{2cm}
  
  \item A car loan for \$20,000 is taken out at a simple interest rate of 7.5\% per year. If the total interest paid over the life of the loan was \$7,500, how many years did it take to pay off the loan?
  \vspace{2cm}
  
  \item Sally wants to have \$2,500 in her savings account after 4 years by earning simple interest at an annual rate of 3\%. How much should she deposit now to achieve this goal?
  \vspace{2cm}
  
  \item A local bank is offering a simple interest rate of 2.5\% per year on a new savings account. If Tom deposits \$800, how much total interest will he have earned after 18 months?
  \vspace{2cm}
  
  \item Sarah invested a certain amount of money at a simple interest rate of 4\% per year. After 6 months, she earned \$48 in interest. How much did she initially invest?
  \vspace{2cm}
  
  \item An investment offers a 5\% simple interest rate per month. If an investor wants to earn \$600 in interest in 4 months, what amount does he need to invest?
  \vspace{2cm}
\end{enumerate}

\subsection*{Answers}
\begin{enumerate}
  \item Using the simple interest formula \( I = P \times r \times t \):
  \begin{align*}
  60 &= 500 \times 0.04 \times t \\
  t &= \frac{60}{500 \times 0.04} \\
  t &= 3 \text{ years}
  \end{align*}
  \vspace{1cm}
  
  \item \( I = P \times r \times t \):
  \begin{align*}
  I &= 1000 \times 0.05 \times 3 \\
  I &= 150 \text{ dollars}
  \end{align*}
  \vspace{1cm}
  
  \item \( I = P \times r \times t \):
  \begin{align*}
  300 &= 1000 \times 0.06 \times 5 \\
  \text{The bond was sold at face value.}
  \end{align*}
  \vspace{1cm}
  
  \item \( I = P \times r \times t \):
  \begin{align*}
  7500 &= 20000 \times 0.075 \times t \\
  t &= \frac{7500}{20000 \times 0.075} \\
  t &= 5 \text{ years}
  \end{align*}
  \vspace{1cm}
  
  \item \( A = P + I \):
  \begin{align*}
  2500 &= P + P \times 0.03 \times 4 \\
  2500 &= P(1 + 0.03 \times 4) \\
  P &= \frac{2500}{1 + 0.12} \\
  P &\approx 2232.14 \text{ dollars}
  \end{align*}
  \vspace{1cm}
  
  \item \( I = P \times r \times t \) (note \( t \) in years):
  \begin{align*}
  I &= 800 \times 0.025 \times 1.5 \\
  I &= 30 \text{ dollars}
  \end{align*}
  \vspace{1cm}
  
  \item \( I = P \times r \times t \) (note \( t \) in years):
  \begin{align*}
  48 &= P \times 0.04 \times 0.5 \\
  P &= \frac{48}{0.02} \\
  P &= 2400 \text{ dollars}
  \end{align*}
  \vspace{1cm}
  
  \item \( I = P \times r \times t \):
  \begin{align*}
  600 &= P \times 0.05 \times 4 \\
  P &= \frac{600}{0.2} \\
  P &= 3000 \text{ dollars}
  \end{align*}
\end{enumerate}

\end{document}
