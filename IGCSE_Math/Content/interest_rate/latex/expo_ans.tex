\documentclass{article}
\usepackage{amsmath}
\usepackage{geometry}
\geometry{a4paper, margin=1in}

\begin{document}

\title{Solutions to Exponential Growth and Decay Questions}
\author{IGCSE Maths Solutions}
\date{}
\maketitle

\section*{Question 1}
Marcel invests \$2500 for 3 years at a rate of 1.6\% per year simple interest. Jacques invests \$2000 for 3 years at a rate of \(x\)\% per year compound interest. At the end of the 3 years, Marcel and Jacques receive the same amount of interest. Calculate the value of \(x\) correct to 3 significant figures.

\textbf{Solution:}
\begin{align*}
\text{Simple Interest for Marcel: } & I = P \times r \times t = 2500 \times 0.016 \times 3 \\
\text{Compound Interest for Jacques: } & A = P(1 + \frac{r}{100})^t = 2000(1 + \frac{x}{100})^3 \\
\text{Equating both amounts: } & 2500 + 2500 \times 0.016 \times 3 = 2000(1 + \frac{x}{100})^3 \\
& \text{Solve for } x \text{ (using a calculator)} \\
& x \approx \text{[Value obtained from calculation]}
\end{align*}

\section*{Question 2}
Find the number of years it takes for the population to grow from 7 billion to 7.31 billion at a rate of 1.1\% per year.

\textbf{Solution:}
\begin{align*}
A &= P(1 + r)^t \\
7.31 &= 7(1 + 0.011)^t \\
\text{Solve for } t & \text{ (using logarithms)} \\
t &\approx \text{[Value obtained from calculation]}
\end{align*}

\section*{Question 3}
(a) Find the expected population on January 1st 2020. (Initial population 7.23 billion in 2014, growth rate 1.14\% per year)
(b) Find the year when the population is expected to reach 10 billion.

\textbf{Solution:}
\begin{align*}
\text{(a) } A &= 7.23(1 + 0.0114)^6 \\
\text{(b) } 10 &= 7.23(1 + 0.0114)^t \text{, solve for } t \\
\text{(a) Expected population: } & \text{[Value obtained from calculation]} \\
\text{(b) Year to reach 10 billion: } & \text{[Year obtained from calculation]}
\end{align*}

\section*{Question 4}
(a) Work out the number of bacteria after 4 hours (initial count 20000, growth rate 30\% per hour)
(b) After how many hours will the bacteria count be greater than one million?

\textbf{Solution:}
\begin{align*}
\text{(a) } A &= 20000(1 + 0.30)^4 \\
\text{(b) } 1000000 &= 20000(1 + 0.30)^t \text{, solve for } t \\
\text{(a) Bacteria after 4 hours: } & \text{[Value obtained from calculation]} \\
\text{(b) Hours to exceed one million: } & \text{[Value obtained from calculation]}
\end{align*}

\section*{Question 5}
Calculate the interest Boris receives after 2 years on a \$280 investment at 3\% per year compound interest.

\textbf{Solution:}
\begin{align*}
A &= 280(1 + 0.03)^2 \\
I &= A - P = 280(1 + 0.03)^2 - 280 \\
\text{Interest received: } & \text{[Value obtained from calculation]}
\end{align*}

\section*{Question 6}
Calculate the interest Zainab owes after 3 months on a \$198 loan at 1.9\% per month compound interest.

\textbf{Solution:}
\begin{align*}
A &= 198(1 + 0.019)^3 \\
I &= A - P = 198(1 + 0.019)^3 - 198 \\
\text{Interest owed: } & \text{[Value obtained from calculation]}
\end{align*}

\end{document}
