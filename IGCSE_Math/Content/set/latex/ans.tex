
\documentclass{article}
\usepackage{amsmath, amssymb}

\begin{document}

\section*{Set Theory Questions and Answers}

\begin{enumerate}
    \item Define a set and give an example of a set that includes three different types of fruits.\\
    \textbf{Answer:} A set is a collection of distinct objects. Example: \( \{ \text{apple, banana, cherry} \} \).\\[1em]

    \item If \( S = \{ a, e, i, o, u \} \), does the letter 'e' belong to set S? Represent your answer using the appropriate set notation.\\
    \textbf{Answer:} Yes, \( e \in S \).\\[1em]

    \item Is the set of all even numbers a finite or infinite set? Explain your reasoning.\\
    \textbf{Answer:} Infinite, because even numbers extend indefinitely.\\[1em]

    \item Consider the sets \( A = \{ 2, 4, 6, 8, 10 \} \) and \( B = \{ 4, 8 \} \). Is set B a subset of set A? Justify your answer using the definition of a subset.\\
    \textbf{Answer:} Yes, \( B \subseteq A \) as all elements of B are in A.\\[1em]

    \item If the universal set \( U \) is the set of all single-digit numbers, and \( A = \{ 1, 2, 3 \} \), what is the complement of set A?\\
    \textbf{Answer:} \( A' = \{ 4, 5, 6, 7, 8, 9 \} \).\\[1em]

    \item Write the set \( \{ 1, 3, 5, 7, 9 \} \) using set-builder notation.\\
    \textbf{Answer:} \( \{ x \mid x \text{ is an odd number less than } 10 \} \).\\[1em]

    \item Given two sets \( A = \{ 1, 2, 3 \} \) and \( B = \{ 3, 4, 5 \} \), find \( A \cap B \) and \( A \cup B \).\\
    \textbf{Answer:} \( A \cap B = \{ 3 \} \), \( A \cup B = \{ 1, 2, 3, 4, 5 \} \).\\[1em]

    \item Give an example of a scenario where the empty set is the correct answer. Describe the scenario and represent the empty set using proper notation.\\
    \textbf{Answer:} The set of cats that are also dogs is an empty set, \( \emptyset \).\\[1em]

    \item Consider a set \( X = \{ 15, 20, 25, 30 \} \). Which elements of set X are divisible by 5? List them in set notation.\\
    \textbf{Answer:} All elements, \( X = \{ 15, 20, 25, 30 \} \).\\[1em]

    \item Are the sets \( A = \{ x \in \mathbb{N} : x \text{ is a prime number less than } 10 \} \) and \( B = \{ 2, 3, 5, 7 \} \) equal? Explain your answer.\\
    \textbf{Answer:} Yes, both sets contain the same prime numbers less than 10.\\[1em]

    \item A \underline{set} is a well-defined collection of distinct objects, considered as an object in its own right. Each object in a set is called an \underline{element}.\\[1em]

    \item If \( A = \{ 2, 4, 6, 8 \} \), then 4 is an \underline{element} of set A.\\[1em]

    \item If every element of set \( B \) is also an element of set \( A \), then \( B \) is a \underline{subset} of \( A \).\\[1em]

    \item If set \( B \) is a subset of set \( A \), and both sets have the exact same elements, then \( A \) and \( B \) are \underline{equal}.\\[1em]

    \item The universal set, often denoted as \underline{\( U \)}, contains every object under consideration, and every set in the context is a \underline{subset} of the universal set.\\[1em]

    \item The set \( \{ x \mid x \text{ is a positive integer less than } 5 \} \) in set-builder notation is written as \underline{\( \{ 1, 2, 3, 4 \} \)}.\\[1em]

    \item The \underline{empty} set, represented by \( \emptyset \), is the set that contains no \underline{elements}.\\[1em]

    \item The notation \( a \in A \) means that \( a \) is an \underline{element} of set \( A \).\\[1em]

    \item A set with a countable number of elements is called a \underline{finite} set, whereas a set with uncountable elements is called an \underline{infinite} set.\\[1em]

    \item In the set of all vowels \( V = \{ a, e, i, o, u \} \), the letter "e" is an \underline{element}, and any consonant is not an \underline{element} of set \( V \).\\[1em]
\end{enumerate}

\end{document}
