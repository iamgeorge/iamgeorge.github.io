\documentclass{article}
\usepackage{amsmath}
\begin{document}

\begin{center}
    \huge TianEn Academy Mathematics 2024 Term 2 Final
\end{center}
\section*{Part 1: Multiple Choice (1 Mark Each)}

\textbf{Question 1:} What is the standard form of a quadratic function?\\
A. \( y = ax + b \) \\
B. \( y = ax^2 + bx + c \) \\
C. \( y = ax^3 + bx + c \) \\
D. \( y = a + b + c \) \\

\vspace{10pt}

\textbf{Question 2:} What is the line of symmetry for the quadratic function \( y = 2x^2 - 4x + 1 \)?\\
A. \( x = 1 \) \\
B. \( x = -1 \) \\
C. \( x = -2 \) \\
D. \( x = 2 \) \\

\vspace{10pt}

\textbf{Question 3:} If a parabola opens downward, what can be said about the coefficient 'a' in the quadratic function \( y = ax^2 + bx + c \)?\\
A. \( a > 0 \) \\
B. \( a < 0 \) \\
C. \( a = 0 \) \\
D. None of the above \\

\vspace{10pt}

\textbf{Question 4:} Which of the following points is the vertex of the quadratic function \( y = -2x^2 + 8x - 3 \)?\\
A. \( (2, 5) \) \\
B. \( (2, -5) \) \\
C. \( (-2, 5) \) \\
D. \( (-2, -5) \) \\

\vspace{10pt}

\textbf{Question 5:} What shape does the graph of a quadratic function take?\\
A. Circle \\
B. Line \\
C. Parabola \\
D. Triangle \\


\textbf{Question 6:} In which quadrant is the point (-3, 2) located? \\
A. Quadrant I \\
B. Quadrant II \\
C. Quadrant III \\
D. Quadrant IV \\

\textbf{Question 7:} What are the coordinates of the origin in the rectangular coordinate system? \\
A. (1, 1) \\
B. (0, 0) \\
C. (-1, 0) \\
D. (0, 1) \\

\textbf{Question 8:} Which of the following points lies on the x-axis? \\
A. (3, 5) \\
B. (-3, 3) \\
C. (0, 4) \\
D. (-4, 0) \\

\textbf{Question 9:} If the equation of a line is \( y = 2x + 3 \), what is the y-coordinate when \( x = 2 \)? \\
A. 4 \\
B. 5 \\
C. 7 \\
D. 8 \\

\textbf{Question 10:} What type of line is represented by the equation \( x = -4 \)? \\
A. Horizontal \\
B. Vertical \\
C. Diagonal \\
D. None of the above \\


\textbf{Question 11:} Which type of graph is best for showing the proportions of a whole? \\
A. Bar Graph \\
B. Line Graph \\
C. Pie Chart \\
D. Histogram \\

\textbf{Question 12:} What type of chart is most suitable for displaying continuous data grouped into intervals? \\
A. Bar Chart \\
B. Pie Chart \\
C. Histogram \\
D. Pictogram \\

\textbf{Question 13:} Which measure of central tendency is most affected by extreme values? \\
A. Mean \\
B. Median \\
C. Mode \\
D. Range \\

\textbf{Question 14:} In a frequency diagram, what does the height of each bar represent? \\
A. The total number of data points \\
B. The frequency of each value \\
C. The range of the data \\
D. The median of the data \\

\textbf{Question 15:} What is the main advantage of using a stem-and-leaf diagram? \\
A. It summarizes the data into a single value \\
B. It shows the spread and preserves the original data \\
C. It displays data in a circular format \\
D. It groups data into ranges \\

\textbf{Question 16:} When calculating the mean from a frequency table, what is the first step? \\
A. Find the total number of data points \\
B. Multiply each value by its frequency \\
C. Sum the frequencies \\
D. Divide the total by the number of values \\

\textbf{Question 17:} Which graph type is most appropriate for showing trends over time? \\
A. Pie Chart \\
B. Bar Chart \\
C. Line Graph \\
D. Scatter Plot \\

\textbf{Question 18:} In a box plot, what does the box represent? \\
A. The interquartile range \\
B. The mean of the data \\
C. The range of the data \\
D. The mode of the data \\

\textbf{Question 19:} What is a key feature of a histogram that differentiates it from a bar chart? \\
A. The bars are of equal width \\
B. The bars can be of different colors \\
C. There are no gaps between the bars \\
D. It displays categorical data \\

\textbf{Question 20:} Which type of graph is best for comparing the frequencies of different categories? \\
A. Line Graph \\
B. Pie Chart \\
C. Scatter Plot \\
D. Bar Chart \\


\textbf{Question 21:} What type of relationship does a scatter graph show? \\
A. Categorical relationship \\
B. Numerical relationship \\
C. Relationship between two sets of data \\
D. Relationship between three sets of data \\

\textbf{Question 22:} Which type of correlation is indicated when the points on a scatter graph form a pattern that slopes upwards from left to right? \\
A. Positive correlation \\
B. Negative correlation \\
C. No correlation \\
D. Constant correlation \\

\textbf{Question 23:} What is the main purpose of a line of best fit in a scatter graph? \\
A. To connect all data points \\
B. To summarize the data \\
C. To estimate results \\
D. To show the range of the data \\

\textbf{Question 24:} In the context of scatter graphs, what does a negative correlation indicate? \\
A. As one variable increases, the other decreases \\
B. Both variables increase together \\
C. Both variables decrease together \\
D. There is no relationship between the variables \\

\textbf{Question 25:} When plotting a scatter graph, what should be plotted on the x-axis? \\
A. Dependent variable \\
B. Independent variable \\
C. Constant variable \\
D. Categorical variable \\

\textbf{Question 26:} What is the main difference between a bar chart and a histogram? \\
A. Bar charts have gaps between the bars; histograms do not \\
B. Histograms have gaps between the bars; bar charts do not \\
C. Bar charts represent continuous data; histograms represent categorical data \\
D. Histograms represent data using circles; bar charts use bars \\

\textbf{Question 27:} In a histogram, what does the area of each bar represent? \\
A. The frequency density \\
B. The class width \\
C. The frequency \\
D. The height of the data \\

\textbf{Question 28:} How is frequency density calculated in a histogram? \\
A. Frequency density = Frequency × Class width \\
B. Frequency density = Frequency + Class width \\
C. Frequency density = Frequency ÷ Class width \\
D. Frequency density = Class width ÷ Frequency \\

\textbf{Question 29:} Why are histograms used instead of bar charts for continuous data? \\
A. Histograms are easier to draw \\
B. Bar charts cannot represent continuous data accurately \\
C. Histograms show gaps between the data \\
D. Bar charts show the frequency density \\

\textbf{Question 30:} What does a histogram help to visualize better compared to a bar chart? \\
A. The mode of the data \\
B. The mean of the data \\
C. The distribution of continuous data \\
D. The median of the data \\


\textbf{Question 31:} What does a cumulative frequency curve show? \\
A. The total frequency at each interval \\
B. The mean of the data \\
C. The mode of the data \\
D. The median of the data \\

\textbf{Question 32:} Which term is also used to describe cumulative frequency? \\
A. Average \\
B. Running total \\
C. Mode \\
D. Range \\

\textbf{Question 33:} What is the first step in plotting a cumulative frequency curve? \\
A. Plotting the median \\
B. Calculating the mean \\
C. Plotting the lowest possible value \\
D. Finding the range \\

\textbf{Question 34:} What is the median value of a data set? \\
A. The highest value \\
B. The middle value \\
C. The most frequent value \\
D. The lowest value \\

\textbf{Question 35:} What is the interquartile range (IQR) used to measure? \\
A. The range of the entire data set \\
B. The spread of the middle 50% of the data \\
C. The difference between the highest and lowest values \\
D. The average of the data set \\

\textbf{Question 36:} Where are the quartiles located in a cumulative frequency curve? \\
A. At 1/3 and 2/3 of the data \\
B. At 1/4 and 3/4 of the data \\
C. At the beginning and end of the data \\
D. At the median and mode of the data \\


\textbf{Question 37:} Which of the following angles are always congruent when two parallel lines are cut by a transversal? \\
A. Same side interior angles \\
B. Corresponding angles \\
C. Same side exterior angles \\
D. Linear pair of angles \\

\textbf{Question 38:} When two parallel lines are cut by a transversal, which of the following pairs of angles are supplementary? \\
A. Alternate interior angles \\
B. Alternate exterior angles \\
C. Same side interior angles \\
D. Corresponding angles \\

\textbf{Question 39:} What is the measure of each alternate interior angle when two parallel lines are cut by a transversal and one of the alternate interior angles is 45°? \\
A. 45° \\
B. 90° \\
C. 135° \\
D. 180° \\

\textbf{Question 40:} If the measure of one of the same side interior angles is 120° when two parallel lines are cut by a transversal, what is the measure of the other same side interior angle? \\
A. 60° \\
B. 120° \\
C. 90° \\
D. 180° \\

\textbf{Question 41:} Which type of angles are formed on the same side of the transversal and outside the two lines? \\
A. Corresponding angles \\
B. Alternate interior angles \\
C. Same side exterior angles \\
D. Vertical angles \\

\textbf{Question 42:} A bag contains 5 red balls, 3 blue balls, and 2 green balls. If one ball is randomly selected from the bag, what is the probability that the ball is red? \\
A. $\frac{5}{10}$ \\
B. $\frac{3}{10}$ \\
C. $\frac{2}{10}$ \\
D. $\frac{1}{4}$ \\

\textbf{Question 43:} In a factory, it is observed that 20 out of every 100 products manufactured are failure. If the factory produces 500 products in a day, what is the predicted number of failure products? \\
A. 50 \\
B. 80 \\
C. 100 \\
D. 120 \\

\textbf{Question 44:} You are planning a lunch menu. You have 4 types of sandwiches (ham, turkey, veggie, and chicken) and 3 types of drinks (water, soda, and juice). According to the Fundamental Counting Principle, how many different lunch combinations can you create? \\
A. 7 \\
B. 10 \\
C. 12 \\
D. 24 \\

\textbf{Question 45:} A box contains 5 red marbles and 3 blue marbles. If you draw one marble, do not replace it, and then draw a second marble, what is the probability of drawing a red marble first and a blue marble second? \\
A. $\frac{5}{64}$ \\
B. $\frac{15}{56}$ \\
C. $\frac{5}{56}$ \\
D. $\frac{3}{56}$ \\

\textbf{Question 46:} A candy company wants to estimate the probability that a randomly chosen piece of candy from a large batch is red. They run a simulation by randomly selecting 1,250 pieces of candy and find that 287 of them are red. Based on this simulation, what is the estimated probability that a randomly chosen piece of candy is red? \\
A. 0.18 \\
B. 0.23 \\
C. 0.26 \\
D. 0.31 \\


\section*{Part 2: Fill-in-the-Blank (2 Marks Each)}

\begin{enumerate}
    \item The graph of a quadratic function is called a \_\_\_\_\_\_\_\_\_\_.
    \item The vertex of a parabola can be found using the formula \( x = \frac{-b}{2a} \) for the \_\_\_\_\_\_\_\_\_\_-coordinate.
    \item For a quadratic function in standard form \( y = ax^2 + bx + c \), the y-coordinate of the vertex can be found by substituting the x-coordinate into the \_\_\_\_\_\_\_\_\_\_.
    \item The line that divides a parabola into two symmetric parts is called the \_\_\_\_\_\_\_\_\_\_.
    \item If the quadratic function \( y = ax^2 + bx + c \) has \( a < 0 \), the parabola opens \_\_\_\_\_\_\_\_\_\_.
    \item The point where the x-axis and y-axis intersect is called the \_\_\_\_\_\_\_\_\_\_.
    \item In the coordinate plane, the coordinates of any point are written as an \_\_\_\_\_\_\_\_\_\_.
    \item The slope of a line is determined by the ratio of the change in \_\_\_\_\_\_\_\_\_\_ to the change in \_\_\_\_\_\_\_\_\_\_.
    \item For the equation of a vertical line, the form is \( x = \_\_\_\_\_\_\_\_\_\_ \).
    \item A set of points that form a straight line are called \_\_\_\_\_\_\_\_\_\_.
    \item A bar chart is most suitable for displaying \_\_\_\_\_\_\_\_\_\_ data.
    \item The median of a data set is the \_\_\_\_\_\_\_\_\_\_ value when the data is ordered.
    \item In a stem-and-leaf diagram, the 'stem' represents the \_\_\_\_\_\_\_\_\_\_ digit(s) of the data.
    \item The mode of a data set is the value that occurs \_\_\_\_\_\_\_\_\_\_.
    \item In a histogram, the area of each bar represents the \_\_\_\_\_\_\_\_\_\_ of the data within that interval.
    \item In a scatter graph, the relationship between two sets of data is shown by plotting \_\_\_\_\_\_\_\_\_\_.
    \item A \_\_\_\_\_\_\_\_\_\_ correlation is indicated when the points on a scatter graph form a pattern that slopes downwards from left to right.
    \item The \_\_\_\_\_\_\_\_\_\_ variable is plotted on the x-axis in a scatter graph.
    \item A line of best fit is used in a scatter graph to \_\_\_\_\_\_\_\_\_\_ results.
    \item When there is no discernible pattern in the points on a scatter graph, it indicates \_\_\_\_\_\_\_\_\_\_ correlation.
    \item In a histogram, the height of each bar represents the \_\_\_\_\_\_\_\_\_\_.
    \item To calculate the frequency density, you divide the frequency by the \_\_\_\_\_\_\_\_\_\_.
    \item The area of each bar in a histogram represents the \_\_\_\_\_\_\_\_\_\_.
    \item Histograms are particularly useful for displaying \_\_\_\_\_\_\_\_\_\_ data.
    \item The class width in a histogram is the difference between the \_\_\_\_\_\_\_\_\_\_ and \_\_\_\_\_\_\_\_\_\_ values in a class interval.
    \item Probability = 1 means the event will \textbf{\_\_\_\_\_\_\_\_\_\_} happen.
    \item Theoretical and experimental probability of an event may or may not be the \_\_\_\_\_\_\_\_\_.
    \item The Fundamental Counting Principle (FCP) is a way to find all the possible outcomes of an event. For example, you are choosing an outfit for the day. You have 3 shirts (red, blue, and green) and 2 pairs of pants (jeans and khakis). The total number of different outfit combinations you can create is \_\_\_\_\_\_\_\_\_\_.
    \item An Independent Event: is an event whose outcome is \_\_\_\_\_\_\_\_\_ affected by another event.
    \item A team of biologists wants to estimate the probability that a certain species of bird will be seen in a specific area. They conduct a simulation by observing the area 200 times and spotting the bird 38 times. Based on this simulation, the estimated probability of seeing the bird in this area is \_\_\_\_\_\_\_\_\_\_.




\end{enumerate}

\end{document}
