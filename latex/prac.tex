\documentclass{beamer}
\usepackage[utf8]{inputenc}
\usepackage{multicol}
\title{The Impact of IB HL Mathematics on Global Student Competence: \\ Pedagogical Insights and Learner Profiles}
\author{George Wang}
\date{}

\begin{document}

\frame{\titlepage}

\begin{frame}
    \frametitle{Abstract}
    \begin{itemize}
        \item \textbf{Exploration:} This research paper explores the impact of IB HL Mathematics on student learning within a global educational framework.
        \item \textbf{Focus:} Identifies distinct advantages of an IB education with a focus on IB Learner Profile attributes (Hayes, 2013).
        \item \textbf{Pedagogical Implications:} Examines curriculum demanding high academic rigor and promoting a global perspective (Chen et al., 2020).
    \end{itemize}
    \vspace{0.3cm}
    \textbf{Key Findings}
    \begin{itemize}
        \item \textbf{Academic Rigor:} Develops strong analytical and problem-solving skills.
        \item \textbf{Global Perspective:} Prepares students for international higher education and careers.
        \item \textbf{Learner Profile Attributes:} Emphasis on inquirers, thinkers, and communicators.
      
    \end{itemize}


    \end{frame}

    \begin{frame}
      \frametitle{Introduction}
      \begin{itemize}
          \item \textbf{Challenging Program:} IB HL Mathematics develops advanced mathematical thinking and problem-solving skills (Sweller, 1988).
          \item \textbf{Beyond Content Knowledge:} Focuses on developing reflective, knowledgeable, and principled learners as defined by the IB Learner Profile.
          \item \textbf{Study Focus:} Assesses how these attributes are fostered and their impact on global student competence.
      \end{itemize}
      \vspace{0.3cm}
      \textbf{Key Points}
      \begin{itemize}
          \item \textbf{Program Design:} Structure and goals of the IB HL Mathematics program.
          \item \textbf{Learner Profile:} Attributes such as reflective, knowledgeable, and principled.
          \item \textbf{Impact on Competence:} Fostering global competence in students.
      \end{itemize}

\end{frame}

\begin{frame}
  \frametitle{Literature Review}
  \begin{itemize}
      \item \textbf{Academic Achievement:} Higher scores in assessments and university studies, especially in STEM fields (Smith, 2019).
      \item \textbf{Inquiry-Based Learning:} Emphasis on inquiry and real-world applications fosters critical thinking and analytical skills.
      \item \textbf{Learner Profile Integration:} Attributes like inquirers, thinkers, and communicators embedded through collaborative and ethical tasks (Jones, 2021).
  \end{itemize}
  \vspace{0.3cm}
  \textbf{Impact on Learners}
  \begin{itemize}
      \item \textbf{Critical Thinking:} Enhanced problem-solving and analytical abilities.
      \item \textbf{Global Awareness:} Broader understanding of global issues and their mathematical contexts.
      \item \textbf{Ethical Reasoning:} Development of ethical decision-making skills through collaborative projects.
  \end{itemize}
  \vspace{0.3cm}
 
\end{frame}

\begin{frame}
  \frametitle{Methodology}
  \begin{itemize}
      \item \textbf{Mixed-Methods Approach:} Combines quantitative data analysis with qualitative interviews and surveys.
      \item \textbf{Quantitative Methods:}
      \begin{itemize}
          \item \textbf{Data Collection:} Gathering performance metrics from international IB schools.
          \item \textbf{Data Analysis:} Statistical analysis to compare academic achievements and performance trends.
      \end{itemize}
      \item \textbf{Qualitative Methods:}
      \begin{itemize}
          \item \textbf{Interviews:} In-depth interviews with IB HL Mathematics students and teachers.
          \item \textbf{Surveys:} Distributing surveys to collect data on cognitive and affective impacts.
      \end{itemize}
  \end{itemize}
  \vspace{0.3cm}
  \textbf{Objectives}
  \begin{itemize}
      \item \textbf{Broader Outcomes:} Understanding cognitive and affective outcomes influenced by the curriculum.
      \item \textbf{Comparative Analysis:} Assessing consistency and impact across different schools and regions.
  \end{itemize}
  \vspace{0.3cm}
  
\end{frame}

\begin{frame}
  \frametitle{Findings and Discussion}
  \begin{itemize}
      \item \textbf{Higher Understanding and Application:} Superior comprehension and application of complex mathematical concepts (Ally, 2008).
      \item \textbf{Skills for Complex Problems:} Preparation to tackle complex and unfamiliar problems valued globally (Baker, 2010).
      \item \textbf{Cultural Awareness:} Enhanced understanding of mathematical concepts from diverse cultural perspectives (Ministry of Education, 2016).
      \item \textbf{Holistic Development:} Promotion of personal and ethical development through the IB Learner Profile (Cornell University Center for Teaching Innovation, 2023).
  \end{itemize}
  \vspace{0.3cm}
\end{frame}
\begin{frame}
  

  \textbf{Discussion}
  \begin{itemize}
      \item \textbf{Global Competence:} Fosters global competence by integrating cultural perspectives.
      \item \textbf{Personal and Ethical Values:} Development through IB Learner Profile attributes.
      \item \textbf{Educational Goals:} Aligns with IB's global educational goals, preparing students for global citizenship.
  \end{itemize}
  \vspace{0.3cm}
  \textbf{Conclusion:} IB HL Mathematics enhances proficiency and prepares students to be culturally competent and ethically aware global citizens.
\end{frame}

\begin{frame}
  \frametitle{Pedagogical Implications}
  \begin{itemize}
      \item \textbf{Academic Rigor with Global Perspective:} Combines high academic standards with a global outlook (Wakhata et al., 2022).
      \item \textbf{Curriculum Design:} Apply IB HL Mathematics insights to enhance other subjects.
      \item \textbf{Inclusive Education:} Promotes inclusivity by integrating diverse cultural perspectives.
  \end{itemize}
  \vspace{0.3cm}
\end{frame}
\begin{frame}
  

  \textbf{Recommendations for Educators}
  \begin{itemize}
      \item \textbf{Interdisciplinary Integration:} Apply inquiry-based learning and global perspectives across subjects.
      \item \textbf{Professional Development:} Ongoing training to incorporate global themes.
      \item \textbf{Collaborative Learning:} Encourage projects addressing global challenges.
  \end{itemize}
  \vspace{0.3cm}
\end{frame}
\begin{frame}
  
  \textbf{Benefits}
  \begin{itemize}
      \item \textbf{Enhanced Engagement:} Engages students in meaningful, real-world problems.
      \item \textbf{Critical Thinking:} Fosters critical thinking through complex tasks.
      \item \textbf{Global Competence:} Prepares globally competent, culturally aware, and ethically responsible citizens.
  \end{itemize}
  \vspace{0.3cm}
  \textbf{Conclusion:} Adopting IB HL Mathematics strategies enhances educational practices across disciplines, fostering a more inclusive and globally aware student body.
\end{frame}



\begin{frame}
  \frametitle{Conclusion}
  \begin{itemize}
      \item \textbf{Comprehensive Education:} IB HL Mathematics integrates rigorous academic standards with holistic development (Zhoc et al., 2019).
      \item \textbf{Global Preparedness:} Prepares students for global challenges with a strong foundation in mathematical skills and global awareness.
      \item \textbf{IB Learner Profile:} Ensures students develop as reflective, knowledgeable, principled, and globally aware individuals (Dimitrov, 2008).
  \end{itemize}
  \vspace{0.3cm}
  \textbf{Implications for Future Education}
  \begin{itemize}
      \item \textbf{Model for Other Disciplines:} The curriculum serves as a model for integrating academic rigor with a global perspective.
      \item \textbf{Holistic Development:} Emphasizes the importance of holistic student development in educational practices.
      \item \textbf{Global Competence:} Highlights the need to prepare students to be globally competent and culturally aware citizens.
  \end{itemize}
  \vspace{0.3cm}
  
\end{frame}


\begin{frame}
  \frametitle{References}
  \begin{multicols}{2}
      \footnotesize{
      \begin{itemize}
          \item Ally, M. (2008). Foundations of Educational Theory for Online Learning. Athabasca University Press. \url{https://eddl.tru.ca/wp-content/uploads/2018/12/01_Anderson_2008-Theory_and_Practice_of_Online_Learning.pdf}
          \item Baker, R. (2010). Data mining for Education. International Encyclopedia of Education, 112-118. \url{https://doi.org/10.1016/b978-0-08-044894-7.01318-x}
          \item Bolin, J. H. (2014). Hayes, Andrew F. (2013). Introduction to mediation, moderation, and conditional process analysis: A regression-based approach. New York, NY: The Guilford press. Journal of Educational Measurement, 51(3), 335-337. \url{https://doi.org/10.1111/jedm.12050}
          \item Cornell University Center for Teaching Innovation. (2023). Measuring student learning. \url{https://teaching.cornell.edu/teaching-resources/assessment-evaluation/measuring-student-learning}
          \item Deci, E., \& Ryan, R. (2000). Commentaries on "The 'What' and 'Why' of goal pursuits: Human needs and the self-determination of behavior". Psychological Inquiry, 11(4), 269-318. \url{https://doi.org/10.1207/s15327965pli1104_02}
      \end{itemize}
      }
  \end{multicols}
\end{frame}

\begin{frame}
  \frametitle{References}
  \begin{multicols}{2}
      \footnotesize{
      \begin{itemize}
          \item Dimitrov, D. M. (2009). Quantitative research in education: Intermediate \& advanced methods.
          \item Findik-Coşkunçay, D., Alkiş, N., \& Özkan-Yildirim, S. (2018). A Structural Model for Students' Adoption of Learning Management Systems. International Forum of Educational Technology \& Society, 21(2), 13-27. \url{https://www.jstor.org/stable/26388376}
          \item Ismail, M., Celebi, E., \& Nadiri, H. (2019). How student information system influence students’ trust and satisfaction towards the University?: An empirical study in a multicultural environment. IEEE Access, 7, 111778-111789. \url{https://doi.org/10.1109/access.2019.2934782}
          \item MOE (Ministry of Education of the People’s Republic of China) (2016b) Notification on plan of 13th five-year plan for ICT in education [in Chinese]. Available at: \url{http://www.moe.edu.cn/srcsite/A16/s3342/201606/t20160622_269367.html}
          \item National Center for Education Statistics. (2021). What does the NAEP mathematics assessment measure? \url{https://nces.ed.gov/nationsreportcard/mathematics/whatmeasure.aspx}
      \end{itemize}
      }
  \end{multicols}
\end{frame}

\begin{frame}
  \frametitle{References}
  \begin{multicols}{2}
      \footnotesize{
      \begin{itemize}
          \item National Research Council, Division of Behavioral and Social Sciences and Education, Center for Education, Board on Testing and Assessment, \& Committee on the Foundations of Assessment. (2001). Knowing what students know: The science and design of educational assessment. National Academies Press.
          \item Pérez-Suay, A., Ferrís-Castell, R., Van Vaerenbergh, S., \& Pascual-Venteo, A. B. (2023). Assessing the relevance of information sources for modelling student performance in a higher mathematics education course. Education Sciences, 13(6), 555. \url{https://doi.org/10.3390/educsci13060555}
          \item SAYGILI, H., \& ÇETİN, H. (2021). The effects of learning management systems (LMS) on mathematics achievement: A meta-analysis study. Necatibey Eğitim Fakültesi Elektronik Fen ve Matematik Eğitimi Dergisi, 15(2), 341-362. \url{https://doi.org/10.17522/balikesirnef.1026534}
          \item Sweller, J. (1988). Cognitive load during problem solving: Effects on learning. Cognitive Science, 12(2), 257-285. \url{https://doi.org/10.1207/s15516709cog1202_4}
          \item Wakhata, R., Mutarutinya, V., \& Balimuttajjo, S. (2022). Secondary school students’ attitude towards mathematics word problems. Humanities and Social Sciences Communications, 9(1). \url{https://doi.org/10.1057/s41599-022-01449-1}
          \item Wang, Y., Liu, X., \& Zhang, Z. (2018). An overview of e-learning in China: History, challenges and opportunities. Research in Comparative and International Education, 13(1). \url{https://doi.org/10.1177/17454999187634}
      \end{itemize}
      }
  \end{multicols}
\end{frame}
\end{document}
